\documentclass{article}

% American Mathematical Society
\usepackage{amsmath}

% For \geometry{}
\usepackage{geometry}

% Set page margins
\geometry{a4paper, scale=0.8}

% For \oiint
\usepackage{esint}

% For indentation of the first line
\usepackage{indentfirst}

% Set title
\title{Maxwell's Equations Notes}
\author{Qingyu Chen}
\date{\today}

\begin{document}
	% set the font size to large
	\large
	% set the line spacing to 2em
	\setlength{\baselineskip}{2em}
	
	% dx or d[#1]
	\renewcommand{\d}[1][x]{\ \text{d}#1}
	
	% basic symbols
	\newcommand{\Time}{t}
	\newcommand{\Line}{L}
	\newcommand{\Surface}{S}
	
	\newcommand{\opDivergence}{\nabla \cdot}
	\newcommand{\opCurl}{\nabla \times}
	
	\newcommand{\ElectricCharge}{Q}
	\newcommand{\ElectricChargeDensity}{\rho_E}
	\newcommand{\MagneticChargeDensity}{\rho_M}
	\newcommand{\ElectricCurrent}{I}
	\newcommand{\ElectricCurrentDensity}{\vec{J}_E}
	\newcommand{\MagneticCurrentDensity}{\vec{J}_M}
	\newcommand{\ElectricField}{\vec{E}}
	\newcommand{\MagneticField}{\vec{H}}
	\newcommand{\ElectricFluxDensity}{\vec{D}}
	\newcommand{\MagneticFluxDensity}{\vec{B}}
	
	\newcommand{\Permittivity}{\varepsilon}
	\newcommand{\Permeability}{\mu}
	\newcommand{\Conductivity}{\sigma}
	
	% make title
	\maketitle
	
	% make table of contents
	\tableofcontents
	
	% new page
	\newpage
	
	\section{Basic Concepts}
	
	\begin{align*}
		\Time &: \text{Time} \ (s)
		\\
		\Line &: \text{Line} \ (m)
		\\
		\Surface &: \text{Surface} \ (m^2)
		\\
		\\
		\opDivergence &: \text{The Divergence Operator}
		\\
		\opCurl &: \text{The Curl Operator}
		\\
		\\
		\ElectricCharge &: \text{Electric Charge} \ (C)
		\\
		\ElectricChargeDensity &: \text{Electric Charge Density} \ (C/m^3)
		\\
		\MagneticChargeDensity &: \text{Magnetic Charge Density (Fictitious)} \ (Wb/m^3)
		\\
		\ElectricCurrent &: \text{Electric Current} \ (C/s = A)
		\\
		\ElectricCurrentDensity &: \text{Electric Current Density} \ (A/m^2)
		\\
		\MagneticCurrentDensity &: \text{Magnetic Current Density (Fictitious)} \ (V/m^2)
		\\
		\ElectricField &: \text{Electric Field} \ (V/m = N/C)
		\\
		\MagneticField &: \text{Magnetic Field} \ (A/m)
		\\
		\ElectricFluxDensity &: \text{Electric Flux Density} \ (C/m^2)
		\\
		\MagneticFluxDensity &: \text{Magnetic Flux Density} \ (Wb/m^2 = T)
		\\
		\\
		\Permittivity &: \text{Permittivity} \ (F/m)
		\\
		\Permeability &: \text{Permeability} \ (H/m)
		\\
		\Conductivity &: \text{Conductivity} \ (S/m)
	\end{align*}
	
	\section{Basic Formula}
	
	\begin{align*}
		\ElectricFluxDensity & = \Permittivity \ElectricField
		\\
		\MagneticFluxDensity & = \Permeability \MagneticField
		\\
		\ElectricCurrentDensity & = \Conductivity \ElectricField
		\\
		\\
		|\ElectricField| & = \frac{\ElectricCharge}{4 \pi \Permittivity r^2}
		\\
		|\MagneticField| & = \frac{\ElectricCurrent}{2 \pi r}
		\\
		\ElectricCurrent & = \int_\Surface \ElectricCurrentDensity \d[\Surface]
	\end{align*}
	
	\section{Maxwell's Equations}
	
	\subsection{Point Form}
	The equations are known as ``point form'' because each equality is true at every point in space.
	\begin{align*}
		\opDivergence \ElectricFluxDensity & = \ElectricChargeDensity
		\\
		\opDivergence \MagneticFluxDensity & = 0
		\\
		\opCurl \ElectricField & = - \frac{\partial \MagneticFluxDensity}{\partial \Time}
		\\
		\opCurl \MagneticField & = \frac{\partial \ElectricFluxDensity}{\partial \Time} + \ElectricCurrentDensity
	\end{align*}
	
	\subsection{Integral Form}
	If the point form of Maxwell's Equations are true at every point, then integrate them over any surface and they will still be true.
	
	Note that in the first two equations, the surface S is a closed surface, which means it encloses a 3D volume. In the last two equations, the surface S is an open surface, that has a boundary line L.
	\begin{align*}
		\oiint_\Surface \ElectricFluxDensity \d[\Surface] & = \ElectricCharge
		\\
		\oiint_\Surface \MagneticFluxDensity \d[\Surface] & = 0
		\\
		\oint_\Line \ElectricField \d[\Line] & = - \iint_\Surface \frac{\partial \MagneticFluxDensity}{\partial \Time} \d[\Surface]
		\\
		\oint_\Line \MagneticField \d[\Line] & = \iint_\Surface \frac{\partial \ElectricFluxDensity}{\Time} \d[\Surface] + \ElectricCurrent
	\end{align*}
	
	\subsection{Time-Harmonic Form}
	Fourier Transforms says that every signal in time can be rewritten as the sum of sine curve. Using a little more complex math and specify the time variation in terms of the sum of sine curve written in complex form.
	
	This form shows that how the waves behave if they are oscillating at a frequency, and all waves can be decomposed into the sum of simple oscillating waves.
	\begin{align*}
		\opDivergence \ElectricFluxDensity & = \ElectricChargeDensity
		\\
		\opDivergence \MagneticFluxDensity & = 0
		\\
		\opCurl \ElectricField & = - i \omega \MagneticFluxDensity
		\\
		\opCurl \MagneticField & = i \omega \ElectricFluxDensity + \ElectricCurrentDensity
	\end{align*}
	
	\subsection{Written Only with E and H}
	Use the first three basic formulas can rewrite Maxwell's Equations with only E and H present.
	\begin{align*}
		\opDivergence \ElectricField & = \frac{\ElectricChargeDensity}{\Permittivity}
		\\
		\opDivergence \MagneticField & = 0
		\\
		\opCurl \ElectricField & = - \Permeability \frac{\partial \MagneticField}{\partial \Time}
		\\
		\opCurl \MagneticField & = \Permittivity \frac{\partial \ElectricField}{\partial \Time} + \Conductivity \ElectricField
	\end{align*}
	
	\subsection{Symmetric Form with Magnetic Charge}
	What if someone finds Magnetic Charge? Then we would also have to alter the equations to allow for magnetic current --- the flow of magnetic charge.
	
	This isn't a purely abstract exercise --- some problems in engineering can be solved more simply by assuming a fictitious magnetic charge or magnetic current.
	\begin{align*}
		\opDivergence \ElectricFluxDensity & = \ElectricChargeDensity
		\\
		\opDivergence \MagneticFluxDensity & = \MagneticChargeDensity
		\\
		\opCurl \ElectricField & = - (\frac{\partial \MagneticFluxDensity}{\partial \Time} + \MagneticCurrentDensity)
		\\
		\opCurl \MagneticField & = \frac{\partial \ElectricFluxDensity}{\partial \Time} + \ElectricCurrentDensity
	\end{align*}
	
\end{document}
