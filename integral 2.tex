\documentclass{article}

% American Mathematical Society
\usepackage{amsmath}

% For \geometry{}
\usepackage{geometry}

% Set page margins
\geometry{a4paper, scale=0.9}

% Don't show the header and footer
\pagestyle{empty}

\begin{document}
	% set the font size to large
	\large
	% set the line spacing to 3em
	\setlength{\baselineskip}{3em}
	
	% dx or d[#1]
	\renewcommand{\d}[1][x]{\ \text{d}#1}
	% plus constant
	\renewcommand{\c}{+ C}
	% for shorten \frac
	\newcommand{\f}{\frac}
	
	\begin{align*}
		&\ \int \sqrt{1 + x^2} \d
		\\
		&\ let\ \ x = \tan{t}
		\\
		= &\ I
		\\
		= &\ \int \sec{t} \d[\tan{t}]
		\\
		= &\ \sec{t} \tan{t} - \int \tan{t} \d[\sec{t}]
		\\
		= &\ \sec{t} \tan{t} - \int \tan^2{t} \sec{t} \d[t]
		\\
		= &\ \sec{t} \tan{t} - \int (\sec^2{t} - 1) \sec{t} \d[t]
		\\
		= &\ \sec{t} \tan{t} - \int \sec^3{t} \d[t] + \int \sec{t} \d[t]
		\\
		= &\ \sec{t} \tan{t} - I + \ln{|\sec{t} + \tan{t}|} \c
		\\
		\\
		\Rightarrow I = &\ \f{1}{2} (\sec{t} \tan{t} + \ln{|\sec{t} + \tan{t}|}) \c
		\\
		= &\ \f{1}{2} (x \sqrt{1 + x^2} + \ln{(x + \sqrt{1 + x^2})}) \c 
	\end{align*}
\end{document}
